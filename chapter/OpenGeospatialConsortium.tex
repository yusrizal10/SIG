\section(Open_Geospatial_Consortium)
\subsection(Defenisi) Informasi Geospasial, yang lazim dikenal dengan peta, adalah informasi
obyek permukaan bumi yang mencakup aspek waktu dan keruangan. Pengertian
geo dalam geospasial, berarti geosfer yang mencakup atmosferlapisan udara
yang meliputi permukaan bumi, litosfer lapisan kulit bumi, pedosfer tanah
beserta pembentukan dan zona-zonanya, sebagai bagian dari kulit bumi,
hidrosfer lapisan air yang menutupi permukaan bumi dalam berbagai bentuknya,
biosfer segenap unsur di permukaan bumi yang membuat kehidupan dan proses
biotik berlangsung dan antroposfer manusia dengan segala aktivitas yang
dilakukannya di permukaan bumi.

\subsubsection(Sejarah)
\cite{lupp2008open} OGC adalah konsorsium industri internasional dari perusahaan, instansi pemerintah, organisasi penelitian, 
dan universitas yang berpartisipasi dalam proses konsensus untuk mengembangkan spesifikasi antarmuka yang tersedia bagi publik. 
Standar OpenGIS mendukung solusi interoperabilitas yang "mengaktifkan geo" layanan Web, nirkabel dan berbasis lokasi, dan arus utama TI. 
Pada awal 1990an, OGC mendefinisikan sebuah visi untuk komputasi geospasial berbasis jaringan. 
Baru-baru ini visi ini telah membuahkan hasil dengan menggunakan layanan web. 
Bagian ini memberikan penglihatan dari tahun 1990an diikuti oleh bagian selanjutnya yang mendefinisikan arsitektur Layanan OGC Web Services. Penerapan komputer dan penggunaan sistem informasi geografis (GIS) secara luas telah menyebabkan peningkatan analisis data geografis dalam banyak disiplin ilmu. Berdasarkan kemajuan teknologi informasi, ketergantungan masyarakat terhadap data tersebut semakin meningkat. Kumpulan data geografis semakin banyak dibagi, dipertukarkan, dan digunakan untuk tujuan
selain yang diinginkan produsen mereka. GIS, penginderaan jarak jauh, pemetaan otomatis dan manajemen fasilitas (AM / FM), 
analisis lalu lintas, sistem geopositioning, dan teknologi lainnya untuk Informasi Geografis (GI) memasuki periode integrasi radikal.

\subsection(Definisi)
\cite{lupp2008open} Open Geospatial Consortium (OGC) Web Services(OWS) adalah layanan yang didefinisikan oleh OGC, 
yang memungkinkan semua jenis fungsi geospasial. 
Ini termasuk layanan untuk akses data, tampilan data dan pengolahan data. 
Permintaan OWS didefinisikan dengan menggunakan protokol Hyper Text Transfer Protocol (HTTP) 
dan dikodekan menggunakan struktur keyvalue-pair (KVP) atau Extensible Markup Language (XML). 
OWS yang paling banyak dikenal adalah Web Map Service (WMS).

