\section(Open Geospatial Consortium)

\subsection(Definisi)
\cite{lupp2008open} Open Geospatial Consortium (OGC) Web Services(OWS) adalah layanan yang didefinisikan oleh OGC, yang memungkinkan semua jenis fungsi geospasial. Ini termasuk layanan untuk akses data, tampilan data dan pengolahan data. Permintaan OWS didefinisikan dengan menggunakan protokol Hyper Text Transfer Protocol (HTTP) dan dikodekan menggunakan struktur keyvalue-pair (KVP) atau Extensible Markup Language (XML). OWS yang paling banyak dikenal adalah Web Map Service (WMS).
\geospasial "WMS" Pada tahun 1998, anggota OGC
setuju untuk beralih dari API yang digabungkan dengan erat ke
bekerja dengan lingkungan layanan web.
Standar pertama yang keluar dari
Aktivitas itu adalah Peta Web OGC
Service Interface (WMS) pada tahun 1998.
Yang lainnya mengikuti. Hal ini, pada gilirannya menyebabkan
mengira ada aplikasi lain
area dan domain yang membutuhkan interoperable
definisi dan akses terhadap geospasial
data, seperti industri jasa lokasi
dan lokasi mengaktifkan komunitas pengguna sensor.
\geospasial "pengertian" Informasi Geospasial, yang lazim dikenal dengan peta, adalah informasi
obyek permukaan bumi yang mencakup aspek waktu dan keruangan. Pengertian
geo dalam geospasial, berarti geosfer yang mencakup atmosferlapisan udara
yang meliputi permukaan bumi, litosfer lapisan kulit bumi, pedosfer tanah
beserta pembentukan dan zona-zonanya, sebagai bagian dari kulit bumi,
hidrosfer lapisan air yang menutupi permukaan bumi dalam berbagai bentuknya,
biosfer segenap unsur di permukaan bumi yang membuat kehidupan dan proses
biotik berlangsung dan antroposfer manusia dengan segala aktivitas yang
dilakukannya di permukaan bumi 1